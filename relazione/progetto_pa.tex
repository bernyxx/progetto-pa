\documentclass[a4paper, 11pt]{article}
\usepackage{graphicx}

\title{Relazione Progetto Programmazione Avanzata}
\author{Kevin Bernardi Matr. n.1058019}
\date{}
\begin{document}
\maketitle
\section{Progetto C++}
\subsection{Introduzione}
Il progetto sulla parte di C++ consiste in una piccola applicazione per tenere traccia di squadre, giocatori, allenatori e delle loro statistiche per il popolare videogioco online competitivo \textit{"Counter Strike: Global Offensive"}, noto anche con l'abbreviazione \textit{"CS:GO"} oppure direttamente \textit{"CSGO"}.
In seguito ci si riferirà al videogioco solamente con quest'ultima sigla.

\subsection{Breve panoramica su CS:GO}
CSGO è videogioco competitivo online a squadre di 5 giocatori. Ogni partita si svolge tra due squadre per un massimo di 30 round. 
Ogni partita si divide in 2 tempi da 15 round ciascuno della durata di 1 minuto e 55 secondi.
La squadra che vince un round guadagna 1 punto e la prima che arriva a 16 round vinti si aggiudica la vittoria (alla meglio di 30). Se una squadra vince ogni round del primo tempo (15-0) e il primo round del secondo tempo (16-0) la partita termina. Se entrambe le squadre arrivano a 15 punti (punteggio 15 a 15), si procede ad oltranza con i round supplementari.
All'inizio della partita una squadra ha il ruolo del terrorista (\textbf{T}) mentre l'altra dell'anti-terrorista (\textbf{CT}). I ruoli assunti dalle squadre rimangono fissi per l'intero tempo. Prima del round 16 i ruoli vengono invertiti.
Ogni squadra per vincere un round ed aggiudicarsi un punto deve:
\begin{itemize}
\item La squadra T può vincere il round facendo esplodere una bomba che deve essere piazzata o eliminando gli avversari entro lo scadere del tempo. La bomba può essere piazzata una sola volta per round;

\item La squadra CT per vincere deve evitare che esploda la bomba piazzata dall'altra squadra o eliminando tutti gli avversari prima che piazzino la bomba o disinnescando la bomba una volta piazzata.
\end{itemize}



\end{document}

